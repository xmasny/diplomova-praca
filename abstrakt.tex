Rytmické počítačové hry predstavujú jedinečný žáner, kde hráči interagujú s hrou prostredníctvom preddefinovaných akcií synchronizovaných s hudobným sprievodom. Tieto akcie sú špecifikované v chart súboroch, ktoré sú obvykle vytvárané manuálnym úsilím skúsených umelcov známych ako chart artists. Táto práca sa zaoberá automatizáciou tohto procesu s cieľom demokratizovať tvorbu chartov a umožniť širšej verejnosti prispievať k tvorbe obsahu pre tieto hry.V rámci prehľadu tejto oblasti analyzujeme štruktúry chart súborov v populárnych rytmických hrách a preskúmame existujúce nástroje a literatúru súvisiacu s ich tvorbou. Na základe tejto analýzy navrhujeme a implementujeme algoritmus na automatizované vytváranie grafov, čím skracujeme proces tvorby a otvárame dvere pre nových tvorcov.Experimentálna časť práce sa zameriava na porovnanie kvality automaticky vytvorených chartov so chartami vytvorenými ľudskými umelcami. Na tento účel používame hodnotenia dobrovoľných hráčov, ktorí budú poskytovať spätnú väzbu na základe svojich herných skúseností. Týmto spôsobom hodnotíme efektivitu a presnosť nášho algoritmu voči manuálnej tvorbe.Výsledky tejto práce môžu prispieť k vývoju nových nástrojov na tvorbu obsahu pre rytmické hry a otvárať diskusiu o možnostiach automatizácie v tvorbe herného obsahu v iných žánroch.