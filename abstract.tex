Rhythmic computer games rely on players executing predefined actions synchronized with accompanying music. These actions are specified in chart files, typically created manually by skilled artists known as chart artists. This work addresses the automation of this process to democratize chart creation and enable a broader community to contribute content to these games. Within the overview of this field, we analyze the structures of chart files in popular rhythmic games and explore existing tools and literature related to their creation. Based on this analysis, we propose and implement an algorithm for automated chart generation, streamlining the creation process and opening the door for new creators. The experimental part of the work focuses on comparing the quality of automatically generated charts with those created by human artists. For this purpose, we utilize evaluations from voluntary players who provide feedback based on their gaming experiences. This approach assesses the effectiveness and accuracy of our algorithm compared to manual creation. The results of this work can contribute to the development of new tools for content creation in rhythmic games and initiate discussions on the possibilities of automation in content creation across different gaming genres.